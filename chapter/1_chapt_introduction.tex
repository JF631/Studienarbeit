\chapter{Introduction}
Long jump is an athletic discipline that is renowned for its technical
complexity and the precise movement patterns it demands from athletes.
Even apparently small technical inaccuracies can significantly impact an
athlete's performance.
Moreover, as shown in~\cite{long_jump_dynamics} the forces during the take-off 
phase can reach up to 10~times the athlete's body weight, 
increasing the risk of serious injuries due to technical inaccuracies.
Therefore, it is crucial to understand and continuously improve these movement 
patterns in training.
However, taken the high approach velocity\footnote{around 10~m/s in male 
semi-professional long jump} into account, this can quickly become a difficult 
task.
Especially the take-off phase can be very short and therefore hard to analyze.\\
%TODO: personal motivation

\noindent Professional athletes often employ expensive high speed camera 
systems in combination with body pose markers to capture and analyze every 
single step they make.\\
Yet, this approach comes with some limitations.
Due to their stationary installation, such camera systems are restricted to a
fixed location.
Moreover, they often combine multiple cameras like Murray et al.~\cite{
elite_camera_setup} used for sprint analysis in order to be able to capture the
whole movement from the beginning of the approach until the landing.
This leads to complex post-processing software requirements.
Additionally, fixed markers need to be attached to an athletes body to be able 
to track their body position.\\

\noindent While these methods provide exact and reliable results, they are 
usually not accessible for hobby- and semi-professional athletes.\\
In recent years however the advances in \ac{AI} and especially within the area 
of deep neural network paved the way for analyzing methods that require less 
complex setups.
As of 2023 deep neural networks trained for body pose detection are even used 
in medical applications like gait analysis~\cite{mp_gait_analysis}.
Because of the already extremely high and continuously improving accuracy, 
its application within the area of motion analysis in long jump is treated in 
the scope of this work.\\
A semi-autonomous drone based evaluation tool is newly developed.
It is supposed to offer a portable alternative to address the lack of existing
opportunities in analyzing long jump performances in training.
For this purpose, the drone should autonomously fly next to the athlete 
throughout the whole jump, capturing their motion and therefore allow for a 
complete jump analysis.
The drone itself is based on \ac{FPV} drone hardware.
It is build from scratch using an onboard single-board computer as flight 
control unit responsible for capturing the video.
Additionally, a ground station software is developed to allow for a convenient
jump analysis regarding the overall body pose as well as a fixed set of 
important parameters, i.e.~knee angles, arm angles, hip position. 
The project's source code is available under \texttt{\url{https://github.com/JF631/FLYJUMP}}.
