\chapter{Introduction}
Long jump is an athletic discipline that is renowned for its technical 
complexity and the precise movement patterns it demands from athletes.
Even apparently small technical insecurities can significantly impact an 
athlete's performance.
Therefore it is crucial to understand and continueously improve these movement 
patterns in training.
However, taken the high approach velocity\footnote{around 10~m/s in male semi 
proffessional long jump} into account, this can quickly become a difficult 
task.
Especially the take-off phase can be very short and therefore hard to analyze.\\

\noindent Professional athletes employ expensive high speed camera systems 
together with body pose markers to capture and analyze every single step 
they make.\\
Yet, such techniques come with some limitations.
Due to their stationary installation, such camera systems are restricted to a
fixed location.
Moreover, they often combine multiple cameras in order to be able to capture 
the whole movement from the beginning of the approach until the 
landing.
This again leads to complex post-processing software requirements.
Additionally, fixed markers need to be attached to an athletes body to be able 
to track the body position.\\

\noindent While these mothods provide exact and reliable results, they are 
usually not accessible for hobby- and semi professional athletes.\\
To address this lack of opportunities in analyzing training performances a 
mobile alternative is developed within this work.
It relies on a drone to capture the athlete and later employs a neural network
to analyze the body position throughout the whole jump.
Both, the drone and the evaluation software will be developed within this work.
