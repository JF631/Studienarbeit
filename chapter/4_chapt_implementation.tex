\graphicspath{{./figures/}}
\chapter{Implementation}
This chapter focuses on the overall project's implementation.
It mainly covers the four parts hardware assembly, long jump analysis,
drone control and their consolidation into one \ac{GUI}.

\section{Hardware setup}\label{sec:4_hardware}
In order to capture high-quality video recordings that cover a complete long 
jump, from the first step all the way to the landing, a drone is used to fly
next to the athlete throughout the whole process.
Thus, a drone in form of a quadcopter is built from scratch.
Its control will be integrated seamlessly in the projects' \ac{GUI}.\\
This section introduces the hardware components that are used for building 
this drone as well as its flight control unit.\\
A short outline of the hardware is given in 
\autoref{subsec:4_hardware_selection},
while \autoref{subsec:4_hw_setup} focuses on the overall assembly of the 
selected hardware.

\subsection{Hardware selection}\label{subsec:4_hardware_selection}
Currently, commercial drone hardware on the market is mainly separable into 
the two large areas of fully remote controlled \ac{FPV} hardware and hardware 
for (autonomous) drones that can usually carry more load, e.g.~heavy cameras.
Even though the quadcopter in this project needs to be remotely 
controllable from a ground station pc, it is still more likely to be located 
in the latter one.\\
Generally the hardware was chosen based on the following criteria:
\begin{itemize}
    \item price
    \item compatibility
    \item size
\end{itemize}

\subsection*{Flight Hardware}\label{subsec:4_filght_hardware}
The main hardware that a quadcopter needs to fly will, in the following, be
referred to as \textit{flight hardware}.
This includes frame, motors, rotors, \acp{ESC} and a \ac{PDB}.\\
The main platform on which all drone hardware is mounted, is referred to as
a quadcopter's frame.
As this project's drone does not need to carry any heavy load, such as high 
precision camera systems or other sensors, a rather compact frame would 
theoretically be sufficient.
However, compact frames tend to be less stable compared to larger frame sizes 
which could lead to a lower video recording quality and thus require more 
complex post-processing software.
Moreover, the assembly process on larger frames is more convenient and 
replacing parts is easier.
Additionally, compact frames are most commonly used in areas that demand quick
reaction times for high speed flight maneuvers, e.g.~in drone racing.
This however is not needed in this project's context.\\
Taken the mentioned considerations into account the mid-sizes \textit{Holybro 
S500 V2} frame kit is chosen.
Besides the frame, the kit also includes a landing gear and rotors.
Moreover, the main platform includes a \ac{PDB} to split the battery's power 
equally to all four motors.\\
An overview of all included parts is given in \autoref{fig:4_frame_kit}.
\begin{figure}[!h]
    \centering
    \includegraphics[scale=0.6]{frame-kit.png}
    \caption[Frame kit]{Holybro S500 V2 frame kit}
    \label{fig:4_frame_kit}
\end{figure}
\FloatBarrier
\noindent Besides the frame, motors and compatible \acp{ESC} are crucial 
flight hardware components.
Each motor requires an own \ac{ESC} that translates signals from a flight 
control unit to a voltage and thereby control the motors' rotation speed.
To guarantee compatibility, both components were chosen from Holybro as well
and can be seen in \autoref{fig:motors_and_esc}.
\begin{figure}[!h]
    \begin{subfigure}[b]{0.48\textwidth}
        \includegraphics*[scale=0.15]{motor.jpg}
        \caption{920KV Motor}
        \label{subfig:motor_picture}
    \end{subfigure}
    \hfill
    \begin{subfigure}[b]{0.5\textwidth}
        \includegraphics*[scale=0.15]{esc.jpg}
        \caption{\acl*{ESC}}
        \label{subfig:esc_picture}
    \end{subfigure}
    \caption[Motor and \acs*{ESC}]{Motor (a) and \acs*{ESC} (b)}
    \label{fig:motors_and_esc}
\end{figure}
\FloatBarrier
The drones' motors performance capabilities are defined by the number of 
\ac{RPM} they can perform per 1V input.
As can be seen in \autoref{subfig:motor_picture}, this link between 
rotation speed and input voltage is expressed in the arbitrary unit \textit{KV}.
The chosen motors are capable of rotating with a speed of 920~\ac{RPM} per 1V 
input voltage. 
Put into context, this is a common rotation speed in commercial and hobby 
drone applications.
Racing drones however, operate at motor speeds of up to 3500~KV.

\subsection*{Control Hardware}\label{subsec:4_control_hardware}
In order to perform flight maneuvers with a quadcopter, each motor must be
controllable individually.
The calculation of the correct rotation speeds is generally performed by 
a \textit{flight control unit}.
Usually, it receives directional instructions from a remote control as input,
combines them with many parameters (e.g.~\acs{GPS} position, height over ground,
speed, etc.) and generates a (\acs{PWM}) output signal for each motor.\\
Within this project the flight controller needs to deal with two different 
inputs.
First, the ground station which can be seen as a remote control in this case.
Additionally, the drone should be able to fly autonomously next to an athlete
during their long jump training.
Here, the second input gets important.
The autonomous fly option requires the quadcopter to perform a person 
detection and therefore image processing on-board.
As the flight controller itself is not able to perform such calculations, an
additional \textit{companion computer} is required.
This companion computer will then send directional instructions just like the 
ones from the ground station to the flight controller and thereby control the 
drone.\\
The combination of flight controller and on-board companion computer will in 
the following be called \textit{control hardware}.\\
There are many types of different flight controllers available commercially.
However, most of them are not meant to be used in combination with a companion
computer.\\
Two of the most commonly used flight controllers in autonomous drone projects
are the \textit{PixHawk} and the \textit{Navio2}.
They are often chosen because they both work together seamlessly with a 
companion computer.
The former is a totally independent system which can also operate without any 
supporting computer.
The latter is implemented as \ac{HAT} specifically designed for a Raspberry 
Pi.
Thus, it does not include an own \ac{CPU} but uses the Raspberry Pis's 
resources to perform flight relevant calculations.\\
A detailed comparison between both flight controllers is given in 
\autoref{table:4_flight_cotroller_cmp}.
\begin{table}[h!]
\centering
\begin{tabular}[c]{|p{4cm}||p{4cm}|p{4cm}|}
\hline
\multicolumn{3}{|c|}{Flight Controller Comparison}\\
\hline
Criteria & PixHawk & Navio2\\
\hline
\hline
Processor & ARM Cortex M4 with FPU / 32-bit co-processor & Depends on Raspberry
Pi version\\
\hline
Sensors & ST Micro 16-bti gyroscope, ST Micro 14-bit accelerometer, MEAS
barometer & MPU9250 9DOF IMU, LSM9DS1 9DOF IMU, MS5611 Barometer, U-blox M8N 
Glonass/\acs{GPS}/Beidou\\
\hline
Interfaces & UART, Spektrum DSM, PPM / S.BUS input, I2C, SPI, CAN, USB, 3.3V 
and 6.6V ADC input, 8 \acs{PWM} outputs, 6 Auxiliary outputs & UART, I2C, ADC, PPM / 
S.BUS input, 14 \acs{PWM} outputs\\
\hline
Dimensions\newline(W x H x L) in mm & $50 \times 15.5 \times 81.5$ & $55 \times 65$\\ 
\hline
Other & Failsafe options (e.g. extra power supply, \acs{GPS}, etc.) & None\\
\hline
Price (Eur.) & TBD & TBD\\
\hline
\end{tabular}
\caption[Flight controller comparison]{Comparison between PixHawk and 
Navio2 flight controller.}
\label{table:4_flight_cotroller_cmp}
\end{table}

\noindent As can be seen, both flight controllers offer different interfaces 
to connect additional hardware.
Moreover, both systems include sensors, mostly to gather information about 
drone's current position and inertia.
Here, the Navio2 even offers more sensors, as it already includes a \ac{GPS} 
sensor, while the PixHawk relies on an external one.\\\\
\noindent For this project, the PixHawk was chosen over the Navio2 mainly for 
three reasons.
First, it is on the market for a long time already and thus have a large 
community support.
Secondly, as it is an independent system, a failure of the companion computer 
will not lead to a crash.
Lastly, it allows for a wide range of companion computers, while the Navio2 
can only interoperate with a Raspberry Pi.\\
Furthermore, the mentioned considerations lead to easy rapid prototyping 
approaches, as the drone can be manually flown without a companion computer in
a first implementation.

\subsection{Hardware assembly}\label{subsec:4_hw_setup}
In the following, a high-level overview of the quadcopters' hardware setup is 
given.
The general wiring is shown and explained before a short introduction of some 
important communication protocols is given.

\subsection*{General wiring}
In the following \autoref{fig:4_general_wiring} the general wiring layout is 
shown.
The whole system is powered from one power source only.
\begin{figure}[!h]
    \centering
    \includegraphics[scale=0.5]{general_wiring.pdf}
    \caption[General Wiring]{General wiring of the hardware setup.\\
    Power connections are labeled red, 
    Blue connections are used for \ac{PWM} signals, 
    Green connections are serial connections, 
    orange connections are more specifically serial UART connections.}
    \label{fig:4_general_wiring}
\end{figure}
\FloatBarrier
\noindent This results in some challenges in providing the correct voltage for
each connected device.
In the current setup this task is taken over by three devices.
The \ac{PM} is directly connected to the battery.
It transfers the battery's voltage to the \ac{PDB} and a lower 5V voltage to 
the PixHawk flight controller.
The \ac{PDB} itself is a parallel circuit, thus providing the same voltage 
(battery voltage) to each output.
The third device is a \ac{BEC} which is directly connected 
to the \ac{PDB} and delivers a constant 5V output. 
This can be used to power a companion computer such as a RaspberryPi.\\
All other required peripherals are powered by the PixHawk flight controller 
itself.
The main peripherals used in this project are a telemetry module which is used
for communication with a ground station and a \ac{GPS} module used for 
improving the drones capabilities to follow a defined trajectory, which is
specifically useful for auto-return~and landing features.
Two more peripherals, a buzzer to output audio warning signals and a manual
kill switch which can immediately stop all four motors, are installed mainly 
for safety reasons.\\
The hardware components that actually control the motor rotation speeds, 
the \acp{ESC}, are connected to the \ac{PDB} for power supply as well as to 
the flight controller that calculates the correct rotation speeds based on the
wanted flight maneuvers and outputs a \ac{PWM} signal for each motor.\\
The presented overall wiring is rather complex but allows relying on one power 
source only instead of using multiple power sources for flight hardware and
control hardware including peripherals respectively.
