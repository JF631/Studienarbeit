\thispagestyle{empty}
\vspace*{\fill}
\begin{center}
        \textbf{Abstract}
\end{center}
In this work, a comprehensive, drone based analysis tool for the technical
analysis of athletics long-jump was developed.
It combines a self-constructed drone for recording long-jump footage with the
development of a ground station software for analyzing any pre-recorded
long-jump videos.\\
The analysis process offers insights in some of the most important approach-
and jumping parameters (e.g. takeoff angle, knee- and arm angles).
The drone enables those parameters to be analyzed throughout the whole jump
in detail.\\
The presented work is devided into the analysis software development and the
development and assembly of the drone.\\
First, an analysis pipeline is implemented which calculates the jumping
parameters based on a video input showing one jumping athlete. 
The pipeline relies on Google's machine learning framework Mediapipe to
calculate the jumping parameters based on the recorded athlete's body
position.
The analysis results are stored in a hdf5 file and can be visualized in the
developed analysis software.
Additionally, an automatic takeoff detection is implemented to support a
convenient analysis process.
Hence, the tool can be used for on-field analysis in outdoor long-jump
training.\\
The drone hardware includes a Pixhawk as flight control unit alongside with
a RaspberryPi companion computer to record and live-stream the long-jump video
footage.
The drone is controllable via the ground station software.\\ 
\vfill

\newpage
\thispagestyle{empty}
\begin{otherlanguage}{german}
\vspace*{\fill}
\begin{center}
        \textbf{Zusammenfassung}
\end{center}
Im Rahmen dieser Arbeit wurde ein Drohnengestütztes Analyse-Werkzeug zur
Technikanalyse von Weitsprüngen in der Leichtathletik entwickelt.
Es kombiniert die Entwicklung einer selbstgebauten Drohne zur Aufnahme von
Weitsprungvideos mit der Entwicklung einer Bodenstationssoftware zur Analyse
beliebig aufgezeichneter Weitsprungvideos.\\
Der Analyseprozess bietet Einblicke in einige der wichtigsten Anlauf- und
Sprungparameter (z.B. Absprungwinkel, Knie- und Armwinkel).
Mit der Drohne können diese Parameter während des gesamten Sprunges im Detail
analysiert werden.\\
Die vorliegende Arbeit gliedert sich in die Entwicklung der Analysesoftware
und die Entwicklung und den Zusammenbau der Drohne.\\
Zunächst wird eine Analysepipeline implementiert, die die Sprungparameter auf
der Grundlage einer Video Eingabe, die einen einzelnen springenden Athleten
zeigt.
Die Pipeline stützt sich auf Googles Mediapipe machine learning Framework, um
die Sprungparameter auf der Grundlage der Körperposition des Athleten zu
berechnen.
Die Analyseergebnisse werden in einer hdf5-Datei gespeichert und können
mittels der entwickelten Analysesoftware visualisiert werden.
Zusätzlich wurde eine automatische Absprungerkennung implementiert, um einen
effizientne Analyseprozess zu gewährleisten.
So kann das entwickelte Tool direkt auf dem Sportplatz für die Analyse von
technischen Weitsprungtrainingseinheiten im Freien eingesetzt werden.\\
Die Hardware der Drohne setzt sich zusammen aus einem Pixhawk als
Flugsteuerungseinheit und einem RaspberryPi Computer zur Aufzeichnung und
Live-Übertragung des Videomaterials.
Die Drohne kann zusätzlich über die Bodenstationssoftware gesteuert werden.
\vfill
\end{otherlanguage}

