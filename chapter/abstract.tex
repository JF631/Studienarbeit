\thispagestyle{empty}
\vspace*{\fill}
\begin{center}
        \textbf{Abstract}
\end{center}
In this work, a comprehensive, drone based tool for analyzing the
motion of long-jump athletes was developed.
The presented work combines the development of a self-constructed drone for
recording long-jump footage with the development of a ground station software
which can be used for analyzing any pre-recorded long-jump video.\\
The analysis results offer insights in some of the most important approach-
and jumping parameters (e.g. takeoff angle, knee- and arm angles).
The drone enables those key parameters to be analyzed throughout the whole
jump in detail.\\
This work is devided into the analysis software development and the
development and assembly of the drone.\\
First, an analysis pipeline is implemented which calculates the jumping
parameters based solely on a video input.
To identify joints and key body points, the pipeline utilizes Google's
machine learning framework Mediapipe.
Subsequently, based on the detection results, the jumping parameters are
calculated.
The analysis results are stored in a hdf5 file and can be visualized within
the developed analysis software.
Additionally, an automatic takeoff frame detection is implemented to support a
convenient and quick analysis process.
Hence, the tool can be used for on-field analysis in outdoor long-jump
training.\\
The drone hardware includes a Pixhawk as flight control unit alongside with
a RaspberryPi companion computer to record and live-stream the long-jump video
footage.
The drone is controllable via the ground station software using the MAVLink
protocol.\\ 
\vfill

\newpage
\thispagestyle{empty}
\begin{otherlanguage}{german}
\vspace*{\fill}
\begin{center}
        \textbf{Zusammenfassung}
\end{center}
Im Rahmen dieser Arbeit wurde ein Drohnengestütztes Analyse-Tool zur
Technikanalyse von Weitsprüngen in der Leichtathletik entwickelt.
Die Arbeit kombiniert die Entwicklung einer selbstgebauten Drohne zur Aufnahme
von Weitsprungvideos mit der Entwicklung einer Bodenstationssoftware zur
Analyse beliebiger, zuvor aufgezeichneter Weitsprungvideos.\\
Die Analyseergebnisse bieten Einblicke in einige der wichtigsten Anlauf- und
Sprungparameter (z.B. Absprungwinkel, Knie- und Armwinkel).
Mit Hilfe der Drohne können diese Parameter während des gesamten Sprungs
detailliert analysiert werden.\\
Die vorliegende Arbeit gliedert sich in die Entwicklung der Analysesoftware
und die Entwicklung und den Zusammenbau der Drohne.\\
Zunächst wird eine Analysepipeline implementiert, die die Sprungparameter
auf Grundlage eines Weitsprungvideos berechnet.
Die Pipeline stützt sich hierbei auf Googles machine learning
Framework Mediapipe, welches zum Erkennen und Lokalisieren von Gelenken
verwendet wird.
Die Ergebnisse werden anschließend genutzt, um die Sprungparameter zu
berechnen.
Zudem werden die Analyseergebnisse in einer hdf5-Datei gespeichert und
können mittels der entwickelten Analysesoftware visualisiert werden.
Zusätzlich wurde eine automatische Absprungerkennung implementiert, um einen
effizienten Analyseprozess zu gewährleisten.
So kann das entwickelte Tool direkt auf dem Sportplatz für die Analyse von
technischen Weitsprungtrainingseinheiten eingesetzt werden.\\
Die Hardware der Drohne setzt sich zusammen aus einem Pixhawk als
Flugsteuerungseinheit und einem RaspberryPi Computer zur Aufzeichnung und
Live-Übertragung des Videomaterials.
Die Drohne kann zudem mittels der Bodenstationssoftware und dem MAVLink
Protokoll gesteuert werden.
\vfill
\end{otherlanguage}

