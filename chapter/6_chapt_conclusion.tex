\graphicspath{{./figures/}}
\chapter{Conclusion and outlook}
Finally, a drone based long-jump evaluation tool has been developed.
A drone was built from scratch which allows recording long-jump
videos of athletes during their training.
The drone is fully controllable by a custom implement software running on a
ground station PC.\\
Furthermore, a long-jump analysis software was developed and integrated in the
ground station's software.
It is able to analyse any pre-recorded long-jump video regarding
knee angles, arm angles and takeoff conditions (takeoff position and takeoff
angle).\\\\
\noindent However, in addition to the jumping parameters that are currently
analyzable a variety of analysis parameters can be added to offer an athlete
the best possible training support.
Therefore, the analysis software is implemented highly modular to allow for
convenient extension development.
One of the most important extensions is the calculation of distances in
video frames which allows for deeper analysis options.
Mainly, an athlete's velocity (especially during the takeoff) can be
calculated based on distance measurements.
Moreover, more advanced parameters, such as the jumping height and
stride lengths during the approach can be analyzed.
Distance measurements however, involve the on-field camera calibration
using a calibration target which might limit the flexibilty.\\\\
\noindent Furthermore, currently, only little video pre-processing is
performed in advance of the body key-point detection stage.
This should be considered in more detail, as it heavily influences the
calculated jumping parameters.
Especially (motion) blurred videos should be pre-processed.\\
To further improve the pose detection quality, transfer-learning approaches
should be considered to refine mediapipe's pose detection model specifically
for lomg-jump videos.\\\\
\noindent Moreover, the takeoff detection is currently based on a
computational expensive minimization problem which is not suitable
for longer video sequences.
Thus, machine learning based options (using \acp{CNN}) should be considered to
improve the takeoff detection performance.\\\\
\noindent Taking the drone into consideration, further improvements are
required to guarantee a stable flight in different conditions.
As the drone flies at rather low heights, \ac{GPS} is not suitable as only
position estimate source for the flight controller.
It should be considered using advanced sensors like optical flow sensors in
combination with height sensors allowing for more exact movements and better
video recording conditions.\\
Finally, flight safety improvements should be implemented.
This especially concerns the usage of distanc sonsors to avoid the drone
crashing into obstacles or even persons.         