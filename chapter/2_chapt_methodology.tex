\graphicspath{{./figures/}}
\chapter{Methodology} 
The following chapter provides an overview over the relevant development 
components that are used within this project.
Therefore, the used software packages are introduced before a short outline of
the utilized drone hardware is given.

\section{Software fundamentals}
\label{sec:2_software_fund}
As the main part of this project's software will run on a portable remote 
computer allowing for not only to control the drone but also to perform the 
long jump analysis on video inputs, every software component is chosen to 
demand as little hardware requirements as possible.
Especially no \ac{GPU} is required to run the software.
All image processing is performed using the \ac{CPU} only.
Furthermore, the software is designed to run platform independent.\\

\subsection{Programming Language and why Python}
\label{subsec:2_programming_language}

\subsection{Mediapipe for detecting body poses}
\label{subsec:2_mediapipe_framework}
One of the software's main tasks is to perform a human body pose detection in
videos.
Because this part runs on the remote computer only, it can also handle
pre-recorded videos that should be evaluated.\\
The evaluation itself is performed using the mediapipe framework.
It uses a pre-trained neural network that is able to detect 33 key points in 
body poses.
The network can also be fine-tuned to improve its' accuracy.
Even if this so called \textit{transfer-training} method requires significantly
less training data than training a neural network from scratch, it is not applied
within this project as first test runs already showed accurate results.\\
Furthermore the mediapipe framework does not require any hardware acceleration
and is renown for it's precise output.
Hii et al. even showed in \cite{mp_gait_analysis} that the framework can be 
applied in medical gait analysis applications to replace marker based 
approaches.\\
Mediapipe can deal with multiple input types including videos and live streams
which is ideal for this project.
%here picture of mediapipe body pose key points

\subsection*{Different detection approaches}
%Advantage over OpenPose: (faster, offers 3d output, no hardware acceleration)
