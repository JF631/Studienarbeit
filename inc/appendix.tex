\graphicspath{{./figures/}}
\chapter*{Anhang}
\addcontentsline{toc}{chapter}{Anhang}
\renewcommand{\thesection}{\Alph{section}}

\section{\acl*{pdm}}
\label{sec:app_pdm}
\begin{figure}[htbp]
    \centering
    \includegraphics[width=0.8\textwidth]{pdm_example.png}
    \caption[Beispiel eines \acs*{pdm} Signals]{\acs*{pdm} Signal einer simplen Sinusfunktion }
\end{figure}

\begin{figure}[htbp]
    \centering
    \includegraphics[width=0.6\textwidth]{pdm_bitstream.png}
    \caption[Beispiel eines \acs*{pdm} Bitstreams]{\acs*{pdm} Bitstream der oben dargestellten Sinuskurve}
\end{figure}
\FloatBarrier
\newpage

\section{UML Diagramme}
\label{sec:class_diag_filterstages}

\begin{figure}[h]
    \centering
    \includepdf[pages=-, scale=0.8]{uml_filterstage.pdf}
    \caption[UML Diagramm des FilterStage Interfaces]{UML Diagramm des FilterStage Interfaces}
\end{figure}
\FloatBarrier

\newpage
\begin{figure}[htbp]
    \centering
    \includepdf[pages=-, scale=0.7]{uml_samplefilterstage.pdf}
    \caption[UML der SampleFilterStage]{Klassendiagramm der SampleFilterStage der schnellen Faltung im Frequenzbereich}
\end{figure}

\newpage
\section{Beispiel eines Lookup Puffers}
\label{sec:example_lookup_buffer}
In der \texttt{BitFilterStage} werden bei der ersten Faltung keine echten Berechnungen durchgeführt. 
Stattdessen werden alle möglichen Ergebnisse vorab berechnet und in einem Lookup Puffer als Textur im Grafikspeicher hinterlegt.\\
Betrachtet man beispielsweise den \acs*{fir} Filter  $F = \texttt{[0.5, 1.0, 0.5]}$ der Länge $3$, gibt es $2^3 = 8$ mögliche Faltungsmöglichkeiten mit einem Bitmuster gleicher Länge.
\begin{table}[!h]
    \centering
    \begin{tabular}{ |p{2cm}|p{5cm}|p{5cm}| }
        \hline
        \multicolumn{3}{|c|}{Lookup Table}\\
        \hline
        Dec & Bitmuster \newline $0.5 \ 1.0 \ 0.5$ & Ergebnis\\
        \hline
        0 & $0 \ \ \ \ 0 \ \ \ \ 0$ & $0$\\
        1 & $0 \ \ \ \ 0 \ \ \ \ 1$ & $0.5$\\
        2 & $0 \ \ \ \ 1 \ \ \ \ 0$ & $1.0$\\
        3 & $0 \ \ \ \ 1 \ \ \ \ 1$ & $1.5$\\
        4 & $1 \ \ \ \ 0 \ \ \ \ 0$ & $0.5$\\
        5 & $1 \ \ \ \ 0 \ \ \ \ 1$ & $1.0$\\
        6 & $1 \ \ \ \ 1 \ \ \ \ 0$ & $1.5$\\
        7 & $1 \ \ \ \ 1 \ \ \ \ 1$ & $2.0$\\
        \hline
    \end{tabular}
    \caption{Beispiel eines Lookup Tables für die erste direkte Faltung.}
    \label{tab:lookup_example}
\end{table}

